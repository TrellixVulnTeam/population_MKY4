\documentclass[a4paper, 14pt]{extreport}
\setcounter{tocdepth}{3}% Включить \subsubsection в Table Of Contents
\setcounter{secnumdepth}{3}% Добавить номер \subsubsection в Table Of Contents

\usepackage{multirow}
\usepackage{cmap}					
\usepackage{mathtext} 				
\usepackage[T2A]{fontenc}
\usepackage[utf8]{inputenc}
\usepackage[english,russian]{babel}

\usepackage{amsmath, amssymb, amsfonts}
\usepackage{graphics}
\usepackage{graphicx}
\usepackage{dsfont}
\usepackage{srcltx}

\usepackage[nodisplayskipstretch]{setspace}
\usepackage{setspace}
\onehalfspacing

\usepackage{caption}
\usepackage{subcaption}
\usepackage{titlesec}
\usepackage{cleveref}
\usepackage[tableposition=top]{caption}
\usepackage{enumitem}
\usepackage{float}
\newcommand\abs[1]{\left|#1\right|}

\usepackage[left=30mm, right=15mm, top=20mm, bottom=20mm]{geometry}
%\usepackage{indentfirst}
%\usepackage{setspace}{onehalfspacing}
%\usepackage{setspace}
%\onehalfspacing
%\usepackage[nodisplayskipstretch]{setspace}

%\linespread{1.3} % то же самое, что ^
%\renewcommand{\rmdefault}{ftm} % Times New Roman
% отступ для первого абзаца главы или параграфа
\usepackage{indentfirst}

\numberwithin{equation}{section}
\renewcommand{\theequation}{\thesection.\arabic{equation}}
%Чтобы ссылки на рисунки были вида 1.2, где 1 это номер секции
\numberwithin{figure}{section}
\numberwithin{table}{section}

% чтобы в списке литературы была нумерация не в квадратных скобках (по файлу с нормами)
\makeatletter
\renewcommand*{\@biblabel}[1]{\hfill#1}
\makeatother

\makeatletter
    \AddEnumerateCounter{\asbuk}{\@asbuk}{м)}
\makeatother
\setlist{nolistsep}
\renewcommand{\labelitemi}{-}
\renewcommand{\labelenumi}{\asbuk{enumi})}
\renewcommand{\labelenumii}{\arabic{enumii})}

\renewcommand{\thesection}{\arabic{section}}
\renewcommand{\thesubsection}{\thesection.\arabic{subsection}}

% likechapter - подобие главы, только без нумерации и в верхнем регистре
\newcommand{\empline}{\mbox{}\newline}
\newcommand{\likechapterheading}[1]{
    \begin{center}
    \textbf{\MakeUppercase{#1}}
    \end{center}
    \empline}

\makeatletter
    \renewcommand{\@dotsep}{2}
    \newcommand{\l@likechapter}[2]{{\bfseries\@dottedtocline{0}{0pt}{0pt}{#1}{#2}}}
\makeatother
\newcommand{\likechapter}[1]{
    \likechapterheading{#1}
    \addcontentsline{toc}{likechapter}{\MakeUppercase{#1}}}

\titleformat{\chapter}[display]
    {\filcenter}
    {}
    {0pt}
    {\large\bfseries}{}

\titleformat{\section}
    {\normalsize\bfseries}
    {\thesection}
    {1em}{}

\titleformat{\subsection}
    {\normalsize\bfseries}
    {\thesubsection}
    {1em}{}

\titleformat{\subsubsection}
    {\normalsize\bfseries}
    {\thesubsubsection}
    {1em}{}

\titlespacing*{\likechapter}{0pt}{-30pt}{8pt}
\titlespacing*{\chapter}{0pt}{-30pt}{8pt}
\titlespacing*{\section}{\parindent}{*4}{*4}
\titlespacing*{\subsection}{\parindent}{*4}{*4}
\titlespacing*{\subsubsection}{\parindent}{*4}{*4}

\DeclareCaptionLabelFormat{rightparen}{#2}
\captionsetup[subfigure]{labelformat=rightparen}
%\captionsetup[subfigure]{position=bottom, textfont=normalfont,singlelinecheck=off,justification=raggedright}

\DeclareCaptionLabelFormat{gostfigure}{Рисунок #2}
\DeclareCaptionLabelFormat{gosttable}{Таблица #2}
\DeclareCaptionLabelSeparator{gost}{~---~}
\captionsetup{labelsep=gost}
\captionsetup[figure]{labelformat=gostfigure}
\captionsetup[table]{labelformat=gosttable}
\renewcommand{\thesubfigure}{\asbuk{subfigure}}

\crefname{figure}{}{}
\Crefname{figure}{}{}

\crefname{equation}{}{}
\Crefname{equation}{}{}

\title{Исследование кусочно-гладкой модели нейронной активности: детерминированный и стохастический случаи}
\author{Беляев Александр}

\begin{document}
\def\contentsname{\likechapterheading{СОДЕРЖАНИЕ}} % likechapterheading - отступ 2 строки от заголовка
\def\bibname{\likechapter{Список использованных источников}}
\begin{titlepage}
\begin{center}
\large
\hspace{-15mm}МИНИСТЕРСТВО ОБРАЗОВАНИЯ И НАУКИ РОССИЙСКОЙ ФЕДЕРАЦИИ\\
Федеральное государственное автономное образовательное учреждение\\
высшего образования\\
УРАЛЬСКИЙ ФЕДЕРАЛЬНЫЙ УНИВЕРСИТЕТ\\
имени первого Президента России Б.Н. Ельцина\\
\vspace{8mm}
ИНСТИТУТ ЕСТЕСТВЕННЫХ НАУК И МАТЕМАТИКИ\\
\vspace{15mm}
Кафедра теоретической и математической физики\\%Департамент математики, механики и компьютерных наук\\
\vspace{12mm}
\textbf{ИССЛЕДОВАНИЕ КУСОЧНО-ГЛАДКОЙ МОДЕЛИ НЕЙРОННОЙ АКТИВНОСТИ: ДЕТЕРМИНИРОВАННЫЙ И СТОХАСТИЧЕСКИЙ СЛУЧАИ} \\
\vspace{7mm}
Направление подготовки 02.03.01 «Математика и компьютерные науки»
\end{center}
\vspace{20mm}

\vfill
\begin{center}
 Екатеринбург\\2018 г.
\end{center}
\end{titlepage}

%\newgeometry{left=30mm, top=20mm, right=15mm, bottom=20mm}

%\clearpage
\setcounter{page}{2}

\newpage
%\thispagestyle{empty} % убирает номер страницы внизу
%\likechapterheading{Реферат}

%\noindent Беляев А.В., ИССЛЕДОВАНИЕ КУСОЧНО-ГЛАДКОЙ МОДЕЛИ НЕЙРОННОЙ АКТИВНОСТИ: ДЕТЕРМИНИРОВАННЫЙ И СТОХАСТИЧЕСКИЙ СЛУЧАИ, выпускная квалификационная работа: стр. 35, рис. 26, табл. 1, библ. 11 назв.

%Ключевые слова: модель Рулькова, кусочно-гладкое отображение, нейронная активность, гомоклиническая бифуркация, бифуркация столкновения с границей, стохастические возмущения, функция стохастической чувствительности, индуцированные шумом феномены, межспайковые интервалы.

%В работе рассматривается одномерная модель нейронной активности, задаваемая кусочно-гладким отображением. Проводится классификация возможных режимов, анализируется существование аттракторов детерминированной модели и их устойчивость. Детально описываются бифуркации: гомоклиническая и бифуркация столкновения с границей. Для стохастической модели, используя метод функций стохастической чувствительности, изучается чувствительность аттракторов к внешнему возмущению, а также на основе метода доверительных полос описываются индуцированные шумом стохастические феномены: переходы внутри аттрактора, переходы между аттракторами, генерация большеамплитудных колебаний, возникновение спайков. Изучаются статистические характеристики межспайковых интервалов.

\tableofcontents

\newpage
\likechapter{Обозначения и сокращения}

В данной работе были использованы следующие обозначения и сокращения: \\
$ x_{n} $ --- быстрая переменная, соответствующая потенциалам действия,\\
$ y_{n} $ --- медленная переменная, соответствующая изменению концентрации открытия ионных каналов,\\
$ \alpha $ и $ \beta $ --- управляющие параметры,\\
$ \varepsilon $ --- интенсивность шума,\\
$ \xi_{t} $ --- случайная величина,\\
$ \rho(x, \varepsilon) $ --- функция плотности вероятности,\\
$ w $ --- функция стохастической чувствительности,\\
$ p(x) $ --- плотность,\\
$ m $ --- среднее значение,\\
$ D $ --- дисперсия,\\
$ C_{\nu} $ --- коэффициент вариации,\\
$ \varepsilon^{*} $ --- критическая интенсивность шума,\\
$ \tau $ --- межспайковый интервал,\\
BCB --- border collision bifurcation --- бифуркация столкновения с границей,\\
ФСЧ --- функция стохастической чувствительности,\\
ISI --- inter spike interval --- межспайковый интервал.

\newpage
\likechapter{Введение}

В настоящее время большое внимание исследователей привлекают модели, описывающие динамику нейрона. Активность биологических нейронов является результатом высокоразмерной динамики нелинейных процессов, ответственных за генерацию и взаимодействие различных ионных токов за счет мембранных каналов. Численные исследования такой нейронной активности обычно основаны либо на физиологических, либо на феноменологических моделях. Большинство рассматриваемых феноменологических моделей описываются системой дифференциальных уравнений третьего порядка и выше (модель Ходжкина-Хаксли~\cite{Hodgkin1952}, модель Хиндмарша-Роуза~\cite{Hindmarsh1984}). Такие системы позволяют моделировать сложное поведение нейрона: спайки и бёрсты. С математической точки зрения такое поведение означает переход от равновесия к периодическим и хаотическим режимам.

В то же время для моделирования различных режимов колебаний с использованием дискретных систем можно ограничиться системами меньшей размерности. В этом случае нейронная активность чаще всего может быть описана системой, состоящей по крайней мере из двух временных шкал: быстрая, соответствующая потенциалам действия, и медленная, соответствущая изменению концентрации открытия ионных каналов~\cite{Bashkirtseva2018}, \cite{Bashkirtseva2018b}, \cite {Rulkov2002}, \cite{Rulkov2018}, \cite{ShilnikovRulkov2003}. Также часто изучаются модели, где вторая медленная переменная принимается как константа. Благодаря этому предположению система становится одномерной~\cite{Rulkov2018}, \cite{Bashkirtseva2015}, \cite{Mesbah2014}.

В данной работе исследуется модификация модели Рулькова, предложенная в работе~\cite{Rulkov2002}:
\begin{equation}
	%\begin{array}{l}
    \begin{cases}
		x_{n + 1} = f(x_{n}, y_{n}), \\
		y_{n + 1} = y_{n} - \mu (x_{n} + 1) + \mu \sigma,
    \end{cases}
	%\end{array}
\end{equation}
где $ x_{n} $ --- быстрая переменная, $ y_{n} $ --- медленная переменная. Медленное изменение $ y_{n} $ обусловлено малыми значениями параметра $ \mu = 0.001 $. Параметр $ \sigma $ является управляющим параметром, который используется для выбора режима индивидуального поведения~\cite{ShilnikovRulkov2003}. В работе рассматривается случай, когда $ y_{n} = const = \beta $, и значит, система становится одномерной.

В ходе исследования был проделан следующий анализ данной модели:
\begin{enumerate}
	\item детерминированный анализ:
    \begin{enumerate}
    	\item нахождение равновесий;
        \item классификация возможных режимов;
        \item устойчивость аттракторов;
        \item исследование бифуркаций:
        \begin{itemize}
        	\item гомоклинической;
            \item столкновения с границей.
        \end{itemize}
    \end{enumerate}
    \item стохастический анализ:
    \begin{enumerate}
    	\item чувствительность аттракторов;
        \item стохастические феномены:
        \begin{itemize}
        	\item переходы между элементами аттрактора;
            \item переходы с равновесия на цикл;
            \item переходы с цикла на равновесие;
            \item генерация большеамплитудных колебаний.
        \end{itemize}
        \item статистический анализ межспайковых интервалов.
    \end{enumerate}
\end{enumerate}

\newpage
\likechapter{Основная часть}

	%\newpage
	\section{Детерминированная модель}
	\label{deterministic_model}
	\subsection{Описание модели и равновесий}
	\label{model_description_eqs}
		%Активность биологических нейронов является результатом высокоразмерной динамики нелинейных процессов, ответственных за генерацию и взаимодействие различных ионных токов за счет мембранных каналов. Численные исследования такой нейронной активности обычно основаны либо на реальных канальных, либо на феноменологических моделях. %\vspace{1mm}
		
		%Нейроны можно описывать системой, состоящей по крайней мере из двух временных шкал: быстрая динамика соответствует потенциалам действия и медленная динамика соответствует изменению концентрации открытия ионных каналов. %\vspace{1mm}

		%В работе рассматривается модель, которая строится как дискретное отображение малой размерности, способное генерировать спайки и бёрсты, возбуждаемые на фоне медленных колебаний. %\vspace{1mm}

		%Модель задается двумерным отображением следующего вида:
		%\begin{equation}
	%		\begin{array}{l}
	%			x_{n + 1} = f(x_{n}, y_{n}), \\[3mm]
%				y_{n + 1} = y_{n} - \mu (x_{n} + 1) + \mu \sigma,
	%		\end{array}
	%	\end{equation}
	%	где $ x_{n} $ --- быстрая переменная, $ y_{n} $ --- медленная переменная. Медленное изменение $ y_{n} $ обусловлено малыми значениями параметра $ \mu = 0.001 $. Параметр $ \sigma $ является управляющим параметром, который используется для выбора режима индивидуального поведения~\cite{ShilnikovRulkov2003}. %\vspace{1mm}

		%В данной работе рассматривается случай, когда $ y_{n} = const = \beta $, и значит, система становится одномерной. %\vspace{1mm}
	
		В данном разделе рассматривается детерминированная модель, заданная кусочно-гладким отображением $ f(x) $, имеющим вид аналогично работе~\cite{ShilnikovRulkov2003}:%, отображение примет вид:
		\begin{equation}
		\label{model:eq1}
			x_{n + 1} = f(x_{n}) =
			\begin{cases}
				\frac{\alpha}{1 - x_{n}} + \beta, & x_{n} \leq 0 \\[3mm]
				\alpha + \beta, & 0 < x_{n} < \alpha + \beta \\[3mm]
				-1, & \alpha + \beta \leq x_{n}
			\end{cases}
		\end{equation} %\vspace{1mm}

        Здесь $ \alpha $ и $ \beta $ --- управляющие параметры отображения.

		Как известно, равновесия одномерного отображения удовлетворяют условию $ f(\bar{x}) = \bar{x} $. Таким образом, уравнение (\ref{model:eq1})  имеет два равновесия:
		\begin{equation}
			\begin{array}{l}
				\bar{x}_{1} = \frac{1}{2} \left(1-\sqrt{-4 \alpha+(-1+\beta)^2}+\beta\right), \\[3mm]
				\bar{x}_{2} =  \frac{1}{2} \left(1+\sqrt{-4 \alpha+(-1+\beta)^2}+\beta\right).
			\end{array}
		\end{equation} %\vspace{1mm}
		В силу кусочно-гладкого вида отображения равновесия могут принимать только отрицательные значения. Таким образом, равновесия существуют, если выполняются условия: $ \alpha > 1 $ и $- \alpha < \beta < 1 - 2 \sqrt{\alpha} $. Если $ \beta = 1 - 2 \sqrt{\alpha} $ и $ \alpha > 1 $, то уравнение (\ref{model:eq1}) имеет единственное отрицательное равновесие: $ \bar{x} = \frac{1}{2} (1 + \beta) $.% \vspace{1mm}

	%\newpage
	\subsection{Классификация возможных режимов}
	\label{equilibriums_location}
		Как отмечалось в предыдущем разделе, равновесия системы (\ref{model:eq1}) могут быть только отрицательными, т.е. получены из условия пересечения прямой $ y = x $ с единственной негоризонтальной веткой, задаваемой уравнением $ y = \frac{\alpha}{1 - x} + \beta $, при $ x \leq 0 $. %\vspace{1mm}

		\begin{figure}[h!]
			\begin{center}
				\includegraphics[width=0.75\textwidth]{fig1.png}
			\end{center}
			\caption{Типичные случаи взаимного расположения графиков функций $ y = f(x) $ и $ y = x $. Значения параметров: (1) $ \alpha = 4, \beta = -2 $; \\(2) $ \alpha = 8, \beta = 1 - 4 \sqrt{2} $; (3) $ \alpha = 4, \beta = -3 $; (4) $ \alpha = 2, \beta = 1 - 2 \sqrt{2} $; \\(5) $ \alpha = 12, \beta = -6.2 $; (6) $ \alpha = 8, \beta = -5 $; (7) $ \alpha = 6, \beta = -5 $; \\ (8) $ \alpha = 2.5, \beta = -2.25 $; (9) $ \alpha = 2, \beta = 1.9 $; (10) $ \alpha = 3, \beta = -3 $ }
			\label{fig1}		
		\end{figure}
		На рисунке \ref{fig1} представлены все возможные варианты (10 примеров) взаимного расположения графиков функций $ y = x $ и $ y = f(x) $. Рисунок \ref{fig1} пример (1) изображает случай, когда уравнение (\ref{model:eq1}) не имеет равновесий. Примеры (2) - (4) на рисунке~\ref{fig1} соответствуют существованию одного равновесия, график негоризонтальной ветки отображения касается прямой $ y = x $. Для примера (2) $ \bar{x} < -1 $, для примера (3) $ \bar{x} = -1 $ и для примера (4) $ \bar{x} > -1 $. Разделение этих случаев важно с точки зрения различности реализуемых динамических режимов. Примеры (5) - (9) (см. рисунок \ref{fig1}) представляют все возможные варианты взаимного расположения двух равновесий $ \bar{x}_{1},~\bar{x}_{2} $ и точки $ x = -1 $. Пример (10) (см. рисунок \ref{fig1}) демонстрирует вырожденный случай, когда горизонтальная ветка $ y = \alpha + \beta $ отсутствует (при $ \beta = - \alpha $), и представляется наименее интересным с точки зрения динамики уравнения (\ref{model:eq1}) . %\vspace{1mm}		
		
	\subsection{Критические точки и абсорбирующие области}
	\label{critical_p_absorbing_i}
		При изучении одномерных отображений (гладких или кусочно-гладких, непрерывных или разрывных) важную роль играют так называемые {\it критические} точки, а так же их прямые и обратные образы~\cite{SushkoGardiniAvrutin2016}. Используя такие точки, можно определить границы абсорбирующих интервалов и хаотических аттракторов; они также используются для получения условий возникновения гомоклинических бифуркаций, бифуркаций столкновения с границей и т.д. Следуя работе~\cite{SushkoGardiniAvrutin2016}, ниже приведены основные определения.%\vspace{1mm}
		
		{\bf Определение 1:} Для одномерного непрерывного отображения {\it критические} точки - это локальные экстремумы этого отображения. Для одномерного разрывного отображения, помимо критических точек, связанных с его непрерывными ветвями, предельные значения функции в точках разрыва также называются {\it критическими} точками. %\vspace{1mm}

		Для изучения динамики одномерного отображения $ f: I \to I, I \subset 	\mathbb{R} $, также определяются интервалы, на которых ограничено асимптотическое поведение и которыми можно ограничить анализ. В этом контексте полезно использовать понятие {\it абсорбирующего} интервала. %\vspace{1mm}

		{\bf Определение 2:} Интервал $ J $ называется {\it абсорбирующим}, если:
		\begin{itemize}
			\item $ f(J) \subseteq J $, т.е. либо $J$ - инвариант ($f(J) = J$), либо $J$ строго отображается в себя ($f(J) \subset J$);
			\item существует окрестность $U$ множества $J$ такая, что для любого $ x \in U $ существует конечное натуральное число $k > 0$, для которого выполняется $ f^k(x) \in J $ (т.е. любая точка $x \in U$ отображается внутрь $J$ за конечное число итераций);
			\item $J$ ограничен двумя различными критическими точками или критической точкой и ее образом.
		\end{itemize} %\vspace{1mm}

		\begin{figure}[h!]
			\begin{center}
				\includegraphics[width=0.75\textwidth]{fig2.png}
			\end{center}
			\caption{Критические точки и абсорбирующая область уравнения (\ref{model:eq1}) для $ \alpha =~10.5,~\beta = -6 $}
			\label{fig2}		
		\end{figure}
		Для кусочно-гладкого отображения (\ref{model:eq1}) существуют две точки, где оно теряет гладкость: $ x = c^{1}_{-1} $ и $ x = c^{2}_{-1} $ (см рисунок \ref{fig2}). Согласно определению 1 существуют три критические точки: $ c^{1} = f(c^{1}_{-1}) $, $ c^{2} = \lim\limits_{x \to -(\alpha + \beta)} f(c^{2}_{-1}) $, $ c^{3} = \lim\limits_{x \to +(\alpha + \beta)} f(c^{2}_{-1}) $, и других критических точек не существует, т.к. ветка $ y = \frac{\alpha}{1 - x_{n}} + \beta $ не имеет экстремумов. Таким образом, критическими точками являются ординаты точек $ A,~B $ и $ D $~(см. рисунок \ref{fig2}). При этом точки $ A $ и $ B $ имеют равную ординату, т.е. $ c^{1} \equiv c^{2} $, и точка $ D $ имеет ординату~$ c^{3} $. %\vspace{1mm}

		Главной особенностью критических точек является то, что они разделяют множество определения $ I $ на подмножества, точки которых имеют разное количество прообразов отображения $ f $. Для отображения (\ref{model:eq1}) критическая точка $ c^{1} \equiv c^{2} $ разделяет множество на два интервала: любая точка $ x_{0} \in Z_{0} = (c^{1} \equiv c^{2}, \infty) $ не имеет прообраза, тогда как любая точка $ x_{0} \in Z_{1} = (\infty, c^{1} \equiv c^{2}) $ имеет один прообраз. Особенными являются сами критические точки, они одновременно образуют множество $ Z_{\infty} = \{ c^{1} \equiv c^{2}, c^{3} \} $ такое, что любая точка $ x_{0} \in Z_{\infty} $ имеет бесконечное число прообразов (см. рисунок \ref{fig2}). %\vspace{1mm}

		Важными при анализе динамических свойств отображения являются как образы порядка $ i $, определяемые как $ c_{i} = f^{i}(c),~i \geq 1 $, так и прообразы порядка $ i $, т.е. $ c = f^{i}(c_{-i}) $. Точки экстремума функции и точки, где функция теряет гладкость, определяются как $ c_{-1} $. На рисунке \ref{fig2} изображены критические точки, а также их прообразы первого порядка --- $ c^{1}_{-1}, c^{2}_{-1} $. %\vspace{1mm}

		Согласно определению 2 множество $ J = (c^{3},~c^{1} \equiv c^{2}) $ является абсорбирующим интервалом отображения (\ref{model:eq1}) и показан на рисунке \ref{fig2} зелёным цветом. На данном множестве возникает предельный цикл некоторого периода $ k $. В силу специфики вида функции $ f $ существуют циклы периода $ k \geq 3 $, и не существуют хаотические аттракторы. Так же легко заметить, что любой цикл всегда имеет один элемент $ c^{3} \equiv -1 $ и один элемент $ c^{1} \equiv c^{2} $. Пример цикла для $ \alpha = 10.5,~\beta = -6 $ представлен на рисунке \ref{fig2_1}а коричневым цветом. %\vspace{1mm}
		
		\begin{figure}[h!]
			\begin{center}
				\includegraphics[width=0.75\textwidth]{fig2_1.png}
			\end{center}
			\caption{а) $ \alpha = 10.5, \beta = -6 $; б) $ \alpha = 6, \beta = -4.2 $}
			\label{fig2_1}		
		\end{figure}

		На рисунке \ref{fig2_1} сверху представлена динамика системы (\ref{model:eq1}) на лестнице Ламерея, снизу --- временными рядами. На рисунке \ref{fig2_1}а изображена ситуация (пример (5) рисунок \ref{fig1}) сосуществования двух равновесий и цикла, при этом, если начальная точка $ x_{0} \in~(-\infty, \bar{x}_{2}) $, то траектория сходится к равновесию $ \bar{x}_{1} $; если же $ x_{0} \in (\bar{x}_{2}, +\infty) $, то --- к циклу периода 4. При этом цикл лежит ровно в абсорбирующем интервале $ J =  (c^{3}, c^{1} \equiv c^{2}) $. %\vspace{1mm}

		На рисунок \ref{fig2_1}б представлена ситуация (пример (7) рисунок \ref{fig1}), когда система (\ref{model:eq1}) не имеет абсорбирующего интервала ($ f(J) \supset J $, не выполняется пункт (1) определения~2). Для любой точки $ x_{0} \in \mathbb{R} (x_{0} \neq \bar{x}_{2}) $ траектории с течением времени сходятся к равновесию $ \bar{x}_{1} $. Однако природа переходного процесса различна: для $ x_{0} \in~(-\infty, \bar{x}_{2}) \cup (c^{2}_{-1} \equiv~c^{3}_{-1}, +\infty) $ траектории монотонно сходятся к равновесию $ \bar{x}_{1} $, для $ x_{0} \in (\bar{x}_{2}, c^{2}_{-1} \equiv~c^{3}_{-1}) $ траектория сначала движется вверх до значения $ c^{1} \equiv c^{2} $, затем за один шаг переходит в $ c^{3} $ и далее монотонно сходится к равновесию $ \bar{x}_{1} $. %\vspace{1mm}

		Таким образом, с помощью теории критических точек удается найти бассейны притяжения для сосуществующих аттракторов, ограничить область, где разворачивается динамика системы, и определить области начальных точек, для которых переходный процесс будет существенно отличаться. %\vspace{1mm}
	
	\subsection{Устойчивость стационарных точек и циклов}
	\label{attractors_stability}
		Известно, что характеристикой устойчивости равновесий для одномерных отображений являются значения $ f'(\bar{x}_{1}) $ и $ f'(\bar{x}_{2}) $, которые называются {\it характеристическими показателями}~\cite{VasinRyashko2003}. %\vspace{1mm}

		Характеристические показатели равновесий уравнения (\ref{model:eq1}) имеют вид:
		\begin{equation}
			\begin{array}{l}
				\lambda_{1} = f'(\bar{x}_{1}) = \frac{\alpha}{\left(1+\frac{1}{2} \left(-1+\sqrt{-4 \alpha+(-1+\beta)^2}-\beta\right)\right)^2}, \\[3mm]
				\lambda_{2} = f'(\bar{x}_{2}) =  \frac{\alpha}{\left(1+\frac{1}{2} \left(-1-\sqrt{-4 \alpha+(-1+\beta)^2}-\beta\right)\right)^2}.
			\end{array}
		\end{equation}
		В зоне существования равновесий $ \bar{x}_{1} $ и $ \bar{x}_{2} $ всегда выполняются следующие неравенства: $ \abs{\lambda_{1}} < 1 $ и $  \abs{\lambda_{2}} > 1 $. Следовательно, равновесие $ \bar{x}_{1} $ --- устойчивое, а равновесие $ \bar{x}_{2} $ --- неустойчивое на всей области параметров $ \alpha > 1 $ и $- \alpha < \beta < 1 - 2 \sqrt{\alpha} $. %\vspace{1mm}
		
		Для циклов характеристическим показателем является выражение:
		\begin{equation}
			\lambda_{ц} = f'(\bar{x}_{1}) f'(\bar{x}_{2}) f'(\bar{x}_{3}) .. f'(\bar{x}_{n}), \\[1mm]
		\end{equation}
		где $ n $ --- период цикла, $ \bar{x}_{i} $ --- элементы цикла $ (i = 1, 2, \ldots, n)$~\cite{VasinRyashko2003}. %\vspace{1mm}
		
		Для уравнения (\ref{model:eq1}) производная правой части имеет вид:
		\begin{equation}
			f'(x) =
			\begin{cases}
				-\frac{\alpha}{(1 - x)^{2}}, & x \leq 0 \\[3mm]
				0, & x > 0
			\end{cases}
		\end{equation}
		Всегда существуют как минимум два элемента $ \bar{x}_{i} > 0 $ и, следовательно, всегда существуют два значения $ f'(\bar{x}_{i}) = 0 $. Формально, производная в точке разрыва $ x = c^{2}_{-1} $ не существует, но, поскольку производная справа равна производной слева и равна 0, то можно считать эту производную равной 0. И значит, характеристический показатель цикла $ \lambda_{\textup{ц}} = 0 $. %\vspace{1mm}

		Таким образом, если уравнение (\ref{model:eq1}) имеет цикл, то он всегда суперустойчив.

	\subsection{Бифуркации}
	\label{bifurcations}
		\subsubsection{Гомоклиническая бифуркация}
		\label{homoclinic_bif}
			Для определения гомоклинической бифуркации требуется ввести несколько вспомогательных определений~\cite{SushkoGardiniAvrutin2016}. %\vspace{1mm}

			Пусть $ f: I \to I, I \subset \mathbb{R} $ отображение и $ \bar{x} $ его стационарная точка. %\vspace{1mm}

			Устойчивым множеством стационарной точки $ \bar{x} $ является множество точек, стремящихся к $ \bar{x} $ прямыми итерациями (под действием $ f $), а неустойчивым множеством $ \bar{x} $ --- множество точек, стремящихся к $ \bar{x} $ обратными итерациями (под действием $ f^{-1} $). Если функция $ f $ не является обратимой, т.е. существует такое число $ k > 1 $, что $ f^{-1} $ имеет $ k $ монотонных ветвей, например, $ f^{-1}_{1}, f^{-1}_{2}, \ldots, f^{-1}_{k} $, тогда неустойчивое множество $ \bar{x} $ определяется последовательностью точек $ \{ f^{-1}_{j_{i}} \}^{\infty}_{i=1} $, сходящейся к $ \bar{x} $ при $ i \to \infty $, где $ j_{i} \in \{1, 2, \ldots, k\} $. %\vspace{1mm}

			{\bf Определение 3:} {\it Утойчивые и неустойчивые множества} стационарной точки $ \bar{x} $ отображения $ f $ определены, соответственно, как
			\begin{equation}
				W^{s}(\bar{x}) = \left\{ x | \lim_{i \to \infty} f^{i}(x) = \bar{x} \right\}, W^{u}(\bar{x}) = \left\{ x \ne \bar{x}  | \lim_{i \to \infty} f^{-i}_{j_{i}}(x) = \bar{x} \right\}
			\end{equation}
			%\vspace{1mm}

			{\bf Определение 4:} Устойчивые и неустойчивые множества отталкивающей стационарной точки $ \bar{x} $ могут пересекаться, т.е. может существовать точка $ p $, принадлежащая пересечению $ W^{s}(\bar{x}) $ и $ W^{u}(\bar{x}) $. Такая точка называется {\it гомоклинической точкой} точки $ \bar{x} $. %\vspace{1mm}

			Ранее в разделе \ref{attractors_stability} определялась устойчивость равновесий отображения (\ref{model:eq1}). Однако, когда система имеет одно равновесие $ \bar{x}_{1} \equiv \bar{x}_{2} = \bar{x} $, то оно обладает локальной полуустойчивостью, т.е. в некоторой окрестности $ (\bar{x} -~\delta, \bar{x} + \delta) $ траектория, запущенная с $ x_{0} \in (\bar{x} - \delta, \bar{x}) $, сходится к $ \bar{x} $, в то время время как с $ x_{0} \in (\bar{x}, \bar{x} + \delta) $ расходится от $ \bar{x} $. %\vspace{1mm}

			На рисунке \ref{fig3b} (3) представлен случай, когда $ \bar{x} = -1 $ и, несмотря на то, что $ \bar{x} $ --- полуустойчива локально, все траектории сходятся к равновесию $ \bar{x} $. Подобное же поведение продемонстрировано на рисунке \ref{fig3b} (4) для случая, когда $ \bar{x} > -1 $. На рисунке \ref{fig3b} (6) представлен пример двух различных равновесий ($ \bar{x}_{1} $ --- устойчивое и $ \bar{x}_{2} $ --- неустойчивое), когда $ \bar{x}_{2} = -1 $. На этом примере также видно, что точки неустойчивого множества равновесия $ \bar{x}_{2} $ сходятся к нему. %\vspace{1mm}

			Таким образом, на рисунке \ref{fig3b} изображены три возможных сценария возникновения гомоклинической бифуркации равновесия $ \bar{x} $ (примеры (3) и (4)) и равновесия $ \bar{x}_{2} $ (пример (6))~\cite{SushkoGardiniAvrutin2016}. %\vspace{1mm}
			\begin{figure}[h!]
				\begin{center}
					\includegraphics[width=0.75\textwidth]{fig3b.png}
				\end{center}
				\caption{Гомоклиническая бифуркация равновесия: (3) $ \alpha = 4,\\ \beta = -3 $; (4) $ \alpha = 2, \beta = 1 - 2 \sqrt{2} $; (6) $ \alpha = 8, \beta = -5 $}
				\label{fig3b}		
			\end{figure}
				
			Из условий $ \bar{x} = -1, \bar{x} > -1 $ и $ \bar{x}_{2} = -1 $ найдены параметрические линии, соответствующие гомоклиническим бифуркациям данных равновесий:
			\begin{equation}
				\beta =
				\begin{cases}
					1 - 2 \sqrt{\alpha}, & 1 < \alpha < 4, \\[3mm]
					-\frac{\alpha}{2} - 1, & \alpha > 4.
				\end{cases}
			\end{equation} %\vspace{1mm}
			На рисунке \ref{fig3a} в плоскости параметров $ (\alpha, \beta) $ построены:
			\begin{itemize}
				\item $ \beta = -\alpha $ (зеленый) --- граница области определения уравнения (\ref{model:eq1}).
				\item $ \beta = 1 - 2 \sqrt{\alpha} $ (красные сплошная и пунктирная) --- граница существования двух равновесий. %\vspace{1mm}

				при $ \beta > 1 - 2 \sqrt{\alpha} $ равновесий нет; %\vspace{1mm}
				
				при $ \beta = 1 - 2 \sqrt{\alpha} $ одно равновесие $ \bar{x} $; %\vspace{1mm}
				
				при $ \beta < 1 - 2 \sqrt{\alpha} $ (но $ \beta > -\alpha $) два равновесия $ \bar{x}_{1} $ и $ \bar{x}_{2} $.
				\item $ \beta = -\frac{\alpha}{2} - 1 $ (синие сплошная и пунктирная) --- граница, где одно из равновесий равно -1.
			\end{itemize}
			На рисунке \ref{fig3a} сплошная красная линия соответствует гомоклинической бифуркации $ \bar{x} $, а синяя сплошная --- гомоклинической бифуркации равновесия $ \bar{x}_{2} $. %\vspace{1mm}
			\begin{figure}[h!]
				\begin{center}
					\includegraphics[width=0.68\textwidth]{fig3a.png}
				\end{center}
				\caption{Параметрические зоны существования равновесий системы}
				\label{fig3a}		
			\end{figure} %\vspace{1mm}
		%\newpage
		\subsubsection{Бифуркация столкновения с границей}
		\label{bc_bif}
			{\bf Определение 5:} {\it Бифуркация столкновения с границей (BCB)} происходит, когда при изменении параметра некоторое инвариантное множество (неподвижная точка, точка цикла или граница любого инвариантного множества) соприкасается с точкой излома, в которой функция, задающая отображение, меняет свое определение, и после этого имеет место качественное изменение динамики~\cite{GardiniTramontanaSushko2010} (см. \cite{Nusse1992}, \cite{Nusse1995}, \cite{SushkoGardiniAvrutin2016}). %\vspace{1mm}

			Особенность этого отображения в том, что любой цикл всегда находится в состоянии BCB, т.к. одна точка всегда совпадает с точкой разрыва. Дополнительно к этому еще одна BCB может случиться, когда точка цикла пересечётся с точкой излома. %\vspace{1mm}

			\begin{figure}[h!]
				\begin{center}
					\includegraphics[width=0.75\textwidth]{fig4.png}
				\end{center}
				\caption{Бифуркация столкновения с границей: (а) $ \alpha = 3, \beta = -1.4 $; (б) $ \alpha = 3, \beta = -1.5 $; (в) $ \alpha = 3, \beta = -1.55 $; (г) $ \alpha = 8, \beta = -4.87 $; \\(д) $ \alpha = 8, \beta \approx -4.88408 $; (е) $ \alpha = 8, \beta = -4.89 $}
				\label{fig4}		
			\end{figure}
			На рисунке \ref{fig4} изображены два примера бифуркации столкновения с границей, при которой происходит рождение цикла большего периода. Рисунок \ref{fig4} а) показывает состояние системы до бифуркации, в системе (\ref{model:eq1}) существует цикл периода 3, рисунок \ref{fig4} в) --- после бифуркации, --- цикл периода 4. На рисунке \ref{fig4} б) показан момент бифуркации, т.е. $ \bar{x}_{2} = 0 $. Аналогично, на рисунке \ref{fig4} г) - е) показана бифуркация рождения цикла периода 6 из цикла периода 5. %\vspace{1mm}
			
			Таким образом, условием возникновения цикла периода $ n $ из цикла периода $ (n - 1) $ для системы (\ref{model:eq1}) является:
			\begin{equation}
				f^{n - 3}(-1) = 0,~n > 3.
			\end{equation} %\vspace{1mm}

			Ниже представлены аналитически найденные параметрические условия увеличения периода цикла с 3 до 4 и с 4 до 5.
			\begin{equation}
				\begin{array}{l}
					3 \to 4: \beta = -\frac{\alpha}{2} \\[1mm]
					4 \to 5: \beta = \frac{1}{4} (2 - \alpha - \sqrt{4 + 12 \alpha + \alpha^{2}})
				\end{array}
			\end{equation}

			 В силу того, что формулы возникновения циклов более высоких периодов слишком громоздкие, в работе они не приводятся.

			\begin{figure}[h!]
				\begin{center}
					\includegraphics[width=0.75\textwidth]{fig5.png}
				\end{center}
				\caption{Карта динамических режимов системы (\ref{model:eq1})}
				\label{fig5}		
			\end{figure}
			На рисунке \ref{fig5} представлена карта динамических режимов. Здесь в плоскости параметров $ (\alpha, \beta) $ цветом представлены режимы:
			\begin{itemize}
				\item зелёный --- равновесие;
				\item синий --- цикл периода 3;
				\item розовый --- цикл периода 4;
				\item и т.д.
			\end{itemize}
			Черными линиями на данном рисунке представлены линии бифуркаций столкновения с границей, найденные ранее аналитически. При приближении сверху значений параметров к линиям гомоклинических бифуркаций (см. рисунок \ref{fig3a}) период цикла увеличивается на 1 до бесконечности, и одновременно зона параметров цикла заданного периода сужается. %\vspace{1mm}

			%\newpage
			\begin{figure}[h!]
				\begin{center}
					\includegraphics[width=0.75\textwidth]{fig6.png}
				\end{center}
				\caption{Бифуркационная диаграмма для: а) $ \alpha = 3 $; б) $ \alpha = 8 $}
				\label{fig6}		
			\end{figure}
			На рисунке \ref{fig6} показаны бифуркационные диаграммы при изменении параметра $ \beta $ и при двух фиксированных значениях $ \alpha $. На рисунке \ref{fig6} а) для $ \alpha = 3 $ система всегда имеет единственный устойчивый аттрактор --- равновесие или цикл, при увеличении параметра $ \beta $ сначала наблюдается рождение цикла бесконечного периода в точке гомоклинической бифуркации равновесия $ \bar{x}_{2} $, и затем бифуркации уменьшения периода цикла на 1. На рисунке \ref{fig6} б) для $ \alpha = 8 $ существует зона изменения параметра $ \beta $, где сосуществуют два аттрактора: равновесие и цикл. Здесь красным показано неустойчивое равновесие, являющееся границей разделения бассейнов притяжения устойчивых равновесия и цикла. В зоне циклов также наблюдается бифуркация уменьшения периода циклов, рожденных также в точке гомоклинической бифуркации $ \bar{x}_{2} $ с бесконечным периодом. %\vspace{1mm}

			На рисунке \ref{fig7a} изображена зависимость характеристических показателей равновесий и цикла от параметра $ \beta $:
			\begin{itemize}
				\item $ \lambda_{1} $ --- характеристический показатель равновесия $ \bar{x}_{1} $ и он всегда меньше 1;
				\item $ \lambda_{2} $ --- характеристический показатель равновесия $ \bar{x}_{2} $ и он всегда больше 1;
				\item $ \lambda_\textup{ц} $ --- характеристический показатель цикла, который всегда равен 0.
			\end{itemize} %\vspace{1mm}

			%\newpage
            \begin{figure}[h!]
				\begin{center}
					\includegraphics[width=0.75\textwidth]{fig7_new2.png}
				\end{center}
				\caption{Характеристический показатель для: а) $ \alpha = 3 $; б) $ \alpha = 8 $}
				\label{fig7a}		
			\end{figure} %\vspace{1mm}

			Таким образом, в первом разделе данной работы проведен детальный параметрический анализ детерминированного одномерного отображения (\ref{model:eq1}), описывающего динамику нейрона. Подробно описаны аттракторы и возникающие бифуркации.

	\newpage
	\section{Стохастическая модель}
	\label{stochastic_model}
		В данном разделе работы рассматривается дискретная стохастическая система:
		\begin{equation}
		\label{stoch_model:eq1}
			x_{t + 1} = f(x_{t}) + \varepsilon \sigma(x_{t}) \xi_{t}, \\[1mm]
		\end{equation}
		где $ \varepsilon $ --- интенсивность шума, $ \xi_{t} $ --- случайная величина, распределенная по нормальному закону с параметрами (0, 1). %\vspace{1mm}

		На рисунке \ref{fig1_stoch} изображена стохастическая диаграмма при двух различных значениях шума: $ \varepsilon = 0.01~(а) $ и $ \varepsilon = 0.1~(б) $. Под действием случайного шума состояния покидают детерминированный аттрактор и образуют вокруг него пучок. При увеличении интенсивности шума разброс в пучке увеличивается, при этом структура циклов может размываться (см. рисунок \ref{fig1_stoch} а: $ \beta \in (-2.5; -2.3) $, рисунок \ref{fig1_stoch} б: $ \beta \in (-2.5; -1.5) $).
		\begin{figure}[h!]
			\begin{center}
				\includegraphics[width=0.75\textwidth]{fig1_stoch.png}
			\end{center}
			\caption{Стохастическая диаграмма для $ \alpha = 3 $: а) $ \varepsilon = 0.01 $; б) $ \varepsilon = 0.1 $}
			\label{fig1_stoch}		
		\end{figure}
		\subsection{Стохастическая чувствительность аттракторов}
		\label{stoch_attr}
			Для аппроксимации вероятностного распределения случайных состояний вокруг детерминированных аттракторов (равновесия или циклов) можно использовать метод функций стохастической чувствительности (ФСЧ). Далее коротко описана идея, лежащая в основе этого метода, предложенная в работах Ряшко Л.~Б.~\cite{Ryashko2013}, Башкирцевой И.~А.~\cite{Bashkirtseva2015}. Данный метод широко и результативно применялся в работах~\cite{Bashkirtseva2018}, \cite{Bashkirtseva2017}, \cite{Bashkirtseva2014}.
			\subsubsection{Чувствительность равновесия}
			\label{stoch_eq_sensivity}
				Пусть система (\ref{stoch_model:eq1}) при $ \varepsilon = 0 $ имеет экспоненциально устойчивое равновесие $ x_{t} \equiv \bar{x} $. Пусть $ x^{\varepsilon}_{t} $ --- решение уравнения (\ref{stoch_model:eq1}) с начальным условием $ x^{\varepsilon}_{0} = \bar{x} + \varepsilon \nu_{0} $. Переменная
				\begin{equation}
					z_{t} = \lim_{\varepsilon \to 0} \frac{x^{\varepsilon}_{t} - \bar{x}}{\varepsilon}
				\end{equation}
				характеризует чувствительность равновесия $ \bar{x} $ по отношению как к начальным, так и к случайным возмущениям. Для последовательности $ z_{t} $ выполняется:
				\begin{equation}
					z_{t + 1} = a z_{t} + b \xi_{t}, \\[1mm]
				\end{equation}
				где $ a = f'(\bar{x}) $, $ b = \sigma(\bar{x}) $~(для отображения (\ref{stoch_model:eq1}) $ b_{i} = 1 $). Динамика вторых моментов $ \nu_{t} = Ez^{2}_{t} $ определяется следующим уравнением: %\vspace{1mm}
				\begin{equation}
					\nu_{t + 1} = a^{2} \nu_{t} + b^{2}.
				\end{equation}
				Для экспоненциально устойчивого равновесия $ \bar{x} $ имеем $ \abs{a} < 1 $, и $ \nu_{t} $ имеет конечный предел для любого $ \nu_{0} $, т.е.:
				\begin{equation}
					w = \lim_{t \to \infty} \nu_{t} = \frac{b^{2}}{1 - a^2}.
				\end{equation} %\vspace{1mm}
				Для достаточно малого $ \varepsilon $ вероятностное распределение $ x^{\varepsilon}_{t} $ также сходится. Это означает, что система (\ref{stoch_model:eq1}) имеет стационарные распределенные решения $ \bar{x}^{\varepsilon}_{t} $ с функцией плотности вероятности $ \rho(x, \varepsilon) $. Эта функция имеет следующее приближение в форме нормального распределения:
				\begin{equation}
					\rho(x, \varepsilon) \approx \frac{1}{\varepsilon \sqrt{2 \pi w}} \exp{\left(- \frac{(x - \bar{x})^{2}}{2 w \varepsilon^{2}}\right)},
				\end{equation}
				со средним значением, равным $ \bar{x} $, и дисперсией $ D $, равной $ \varepsilon^{2} w $. Значение $ w $ связывает величину входа ($ \varepsilon^{2} $) и выхода ($ D = \varepsilon^{2} w $) в системе (\ref{stoch_model:eq1}), характеризуя стохастическую чувствительность равновесия $ \bar{x} $. Для {\it функции стохастической чувствительности равновесия} $ w $ явная формула выглядит следующим образом:
				\begin{equation}
					w = \frac{\sigma^{2}(\bar{x})}{1 - (f'(\bar{x}))^{2}}.
				\end{equation}
				Так как у системы (\ref{stoch_model:eq1}) $ \sigma(x_{t}) = 1 $ и $ f'(x) = \frac{\alpha}{(1 - x)^{2}} $, тогда:
				\begin{equation}
				\label{wvalue:eq1}
					w = \frac{\left(1-\beta+\sqrt{(\beta-1)^{2}-4\alpha}\right)^{4}}{\left(1-\beta+\sqrt{(\beta-1)^{2}-4 \alpha}\right)^{4}-16\alpha^2}.
				\end{equation} %\vspace{1mm}

			\subsubsection{Чувствительность циклов}
			\label{stoch_cycle_sensivity}
				Пусть система (\ref{stoch_model:eq1}) при $ \varepsilon = 0 $ имеет цикл периода $ k $: $ \{\bar{x}_{1}, \bar{x}_{2}, \ldots, \bar{x}_{k}\} $. Элементы цикла связаны равенствами:
				\begin{equation}
					f(\bar{x}_{i}) = \bar{x}_{i + 1}~(i = 1, \ldots, k - 1),~f(\bar{x}_{k}) = \bar{x}_{1}.
				\end{equation}
				Последовательность $ \bar{x}_{t} $ определена для всех $ t $ в силу периодичности условия $ \bar{x}_{t + k} = \bar{x}_{t} $. Пусть этот цикл экспоненциально устойчив. Это означает, что для него выполняется следующее неравенство:
				\begin{equation}
				\label{cycles:ineq1}
					\abs{a} < 1, \\[1mm]
				\end{equation}
                где $ a = a_{1} \cdot a_{2} \cdot \ldots \cdot a_{k} $ и $ a_{i} = f'(\bar{x}_{i}) $.
				Для асимптотики отклонений состояний $ x^{\varepsilon}_{t} $ системы (\ref{stoch_model:eq1}) от точек $ \bar{x}_{t} $ выполняется:
				\begin{equation}
					z_{t} = \lim_{\varepsilon \to 0} \frac{x^{\varepsilon}_{t} - \bar{x}_{t}}{\varepsilon}.
				\end{equation}
				Тогда можно записать следующее равенство:
				\begin{equation}
					z_{t + 1} = a_{t} z_{t} + b_{t} \xi_{t}, \\[1mm]
				\end{equation}
				где $ a_{t} = f'(\bar{x}_{t}) $, $ b_{t} = \sigma(\bar{x}_{t}) $. Динамика вторых моментов $ \nu_{t} = Ez^{2}_{t} $ определяется следующим уравнением: %\vspace{1mm}
				\begin{equation}
				\label{cycles:eq1}
					\nu_{t + 1} = a^{2}_{t} \nu_{t} + b^{2}_{t}.
				\end{equation}
				В силу неравенства (\ref{cycles:ineq1}) последовательность $ \nu_{t} $ имеет конечный предел:
				\begin{equation}
					\lim_{t \to \infty} (\nu_{t} - w_{t}) = 0, \\[1mm]
				\end{equation}
				где $ w_{t} $ --- это уникальное $ k $-периодическое решение уравнения (\ref{cycles:eq1}). Для $ w_{1} $ явная формула может быть записана следующим образом:
				\begin{equation}
                \label{w1cycle:eq1}
					w_{1} = \frac{b^{2}_{n} + b^{2}_{n - 1} a^{2}_{n} + \ldots + b^{2}_{1} a^{2}_{2} \cdot \ldots \cdot a^{2}_{n}}{1 - a^{2}}.
				\end{equation}
				Остальные значения $ w_{2}, \ldots, w_{k} $ могут быть найдены реккурентно:
				\begin{equation}
				\label{otherval:eq1}
					w_{i} = a^{2}_{i - 1} w_{i - 1} + b^{2}_{i - 1},~(i = 2, \ldots, k).
				\end{equation}
				Значения $ w_{1}, \ldots, w_{k} $ $ k $-периодической функции $ w_{t} $ характеризуют реакцию элементов $ x_{1}, \ldots, x_{k} $ цикла на малые случайные возмущения. Вектор $ w = (w_{1}, \ldots, w_{k})^{T} $ называется {\it функцией стохастической чувствительности цикла}. %\vspace{1mm}

				Функция стационарной плотности вероятности $ \rho(x, \varepsilon) $ случайных состояний системы (\ref{stoch_model:eq1}) вблизи элементов $ x_{1}, \ldots, x_{k} $ цикла имеет приближение в форме нормального распределения:
				\begin{equation}
				\label{rhocycle:eq1}
					\rho(x, \varepsilon) \approx \frac{1}{\varepsilon k \sqrt{2 \pi}} \sum_{i = 1}^{k} \frac{1}{\sqrt{w_{i}}} \exp{\left(-\frac{(x - \bar{x}_{i})^{2}}{2 w_{i} \varepsilon^{2}}\right)}.
				\end{equation}
				Для системы (\ref{stoch_model:eq1}) ФСЧ циклов можно без труда найти аналитически. Ниже записаны их представления для цикла периода 3 и 4:
				\begin{equation}
				\label{otherval:eq2}
					3: \quad w = \left(1,~\frac{\alpha^{2} + 16}{16},~1\right)^{T},
				\end{equation}
				\begin{equation}
				\label{otherval:eq3}
					4: \quad w = \left(1,~\frac{\alpha^{2} + 16}{16},~1+\frac{\alpha^2 \left(16+\alpha^2\right)}{(\alpha+2 \beta-2)^4},~1\right)^{T}.% \\[1mm]
				\end{equation} %\vspace{1mm}
				%\newpage

				На рисунке \ref{fig2_stoch} изображены функции стохастической чувствительности равновесий (чёрный) и циклов (синий) при двух разных значениях параметра $ \alpha $. Здесь наблюдаются два различных поведения: в точке потере устойчивости равновесиями $ w $ стремится к бесконечности и $ w $ циклов стремится к конечному числу в точках бифуркаций столкновения с границей. Значения этих конечных пределов легко найти по формулам (\ref{otherval:eq2}) и (\ref{otherval:eq3}) для циклов периодов 3 и 4 соответственно и по формуле (\ref{otherval:eq1}) для циклов более высоких периодов. Например, конкретные значения ФСЧ цикла периода 4 представлены в таблице \ref{tabular:w_for_4cycle}, где $ \beta $ --- точка бифуркации столкновения с границей. %\vspace{1mm}

				\begin{table}[h]
                \caption{Значение ФСЧ цикла периода 4}
				\label{tabular:w_for_4cycle}
					\begin{center}
						\begin{tabular}{| l | c | c | c | c  |}
							\hline \multirow{2}{*}{Значения параметров}
                             & \multicolumn{4}{c|}{ФСЧ} \\
                             \cline{2-5}
							 & $ w_{1} $ & $ w_{2} $ & $ w_{3} $ & $ w_{4} $ \\
							\hline
							$ \alpha = 3,~\beta = -2 $ & 1 & 1.5625 & 3.78 & 1 \\
							\hline
							$ \alpha = 8,~\beta = -4.702 $ & 1 & 5 & 39.13 & 1 \\
							\hline
						\end{tabular}
					\end{center}
				\end{table}
				%\newpage

				\begin{figure}[h!]
					\begin{center}
						\includegraphics[width=0.75\textwidth]{fig2_stoch_new_r.png}
					\end{center}
					\caption{ФСЧ аттракторов: а) $ \alpha = 3 $; б) $ \alpha = 8 $}
					\label{fig2_stoch}		
				\end{figure} %\vspace{1mm}

				\begin{figure}[h]
					\begin{center}
						\includegraphics[width=0.75\textwidth]{fig3_stoch_new_r.png}
					\end{center}
					\caption{Доверительные интервалы при $ \varepsilon = 0.01 $ и $ \alpha = 3 $ для: \\ а) равновесия; б) циклов периода не выше 11}
					\label{fig3_stoch}		
				\end{figure} %\vspace{1mm}

				Доверительный интервал $ (x^{*}_{1},~x^{*}_{2}) $ для системы (\ref{stoch_model:eq1}) могут быть построены по правилу трех сигм по следующим формулам:
				\begin{equation}
					x^{*}_{1, 2} = \bar{x} \pm 3 \varepsilon \sqrt{w}, \\[1mm]
				\end{equation}
				где $ \bar{x} $ --- равновесие (элемент цикла), $ w $ --- ФСЧ равновесия (соответствующего элемента цикла). %\vspace{1mm}

				На рисунке \ref{fig3_stoch} зелёным цветом представлены доверительные полосы для равновесия (см. рисунок \ref{fig3_stoch}а) и циклов (см. рисунок \ref{fig3_stoch}б), составленные из доверительных интервалов для каждого значения параметра $ \beta $ при $ \alpha = 3 $. Здесь синим цветом изображено устойчивое равновесие детерминированной системы, серым --- случайные состояния для заданной интенсивности шума. Как можно заметить, доверительная полоса хорошо описывает распределение случайных состояний стохастической системы. При приближении к точке потере устойчивости верхняя и нижняя границы интервалов устремляются к $ +\infty $ и $ -\infty $ соответственно. Рисунок \ref{fig3_stoch} б) так же позволяет убедиться в том, что элементы циклов стохастической системы лежат внутри доверительных полос.
				%\newpage

			\subsubsection{Аппроксимация дисперсии и плотности распределения}
            \label{approx_disp_density}
				Функция стохастической чувствительности позволяет построить аппроксимацию дисперсии разброса случайных состояний стохастического аттрактора. %\vspace{1mm}

                Эмпирическая дисперсия для равновесия может быть вычислена по следующей формуле:
				\begin{equation}
					\overline{D} = \frac{1}{N \varepsilon^{2}} \sum\limits_{j=1}^N (x_{j} - \mu)^2, \\[1mm]
				\end{equation}
				где $ N = 10000 $ --- количество случайных состояний для одной реализации, $ \mu $ --- среднее значение элементов выборки. %\vspace{1mm}

                Аналогично, общая формула эмпирической дисперсии для $i$-того элемента цикла периода $ k $~имеет вид:
				\begin{equation}
					\overline{D_{i}} = \frac{1}{m_{i} \varepsilon^{2}} \sum\limits_{j=0}^{m_{i}-1} (x_{i+kj} - \mu_{i})^2, \\[1mm]
				\end{equation}
				где $ m_{i} $ --- количество случайных состояний, полученных для $i$-того элемента цикла, $ \mu_{i} $~---~среднее значение $ m_{i} $ элементов выборки. %\vspace{1mm}

                Теоретическая дисперсия $ D $ связана с функцией стохастической чувствительности следующими соотношениями:
                \begin{equation}
                	\begin{array}{l}
                		D = \varepsilon^2 w,\\
                        D_{i} = \varepsilon^2 w_{i}, \\[1mm]
                	\end{array}
				\end{equation}
                где $ w $ и $ w_{i} $ находятся по формулам (\ref{wvalue:eq1}), (\ref{w1cycle:eq1}) и (\ref{otherval:eq1}). %\vspace{1mm}

				На рисунке \ref{fig_stoch_disp} изображены теоретическая (синий цвет) и эмпирическая (красный) дисперсии распределения случайных состояний. Случай а) --- разброс случайных состояний вокруг устойчивого равновесия. В случае б) показана аппроксимация дисперсии для циклов до периода 7. Хорошо видно соответствие эмпирической дисперсии с теоретической в обоих случаях.	
				\begin{figure}[h!]
					\begin{center}
						\includegraphics[width=0.75\textwidth]{fig_stoch_disp.png}
					\end{center}
					\caption{Аппроксимация дисперсии при $ \alpha = 3 $ и $ \varepsilon = 0.01 $ для: \\а) равновесия; б) циклов до периода 7}
					\label{fig_stoch_disp}		
				\end{figure}

				Как описывалось в предыдущем разделе, ФСЧ также позволяет построить аппроксимацию плотности вероятности случайных состояний системы, подверженной случайному воздействию. %\vspace{1mm}

				Эмпирическая плотность вычисляется по следующей формуле:
				\begin{equation}
					\overline{p}(x) = \frac{\widetilde{p}(x)}{h N}, \\[1mm]
				\end{equation}
				где $ \widetilde{p}(x) $ --- функция, которая возвращает количество точек случайной траектории, принадлежащих заданному специальным образом интервалу, в котором лежит точка $ x $, $ h $~---~шаг, с котором исходный отрезок разбивается на полуинтервалы, $ N $ --- количество итераций. %\vspace{1mm}

                На рисунке \ref{fig4_stoch} для трех значений параметра $ \beta $, соответствующих циклам периодов 3, 4 и 5, представлены:
                \begin{itemize}
                	\item теоретическая плотность распределения $ p(x) $ (синий цвет);
                    \item эмпирическая плотность распределения $ \overline{p}(x) $ (красный цвет).
                \end{itemize} %\vspace{1mm}

                Как можно заметить, эмпирическая плотность в целом хорошо согласуется с теоретической, кроме правого пика, соответствующего элементу цикла с б$\acute{о}$льшим значением координаты $ x_{k} $. Однако легко заметить, что все левые пики эмпирической плотности распределения едва ниже теоретической, в то время как правый пик заметно выше. Подобное поведение объясняется кусочно-гладкой структурой рассматриваемой модели: в точке $ x = \alpha + \beta $ функция, задающая отображение (\ref{model:eq1}), имеет неустранимый разрыв первого рода ("скачок" см. рисунок \ref{fig2}). %\vspace{1mm}

                Итак, в зависимости от знака значения случайной величины $ \xi_{t} $ определяется, каким будет следующее значение:
				\begin{enumerate}
					\item $ x = -1 $, если $ \xi_{t} > 0 $;
                    \item $ x = \alpha + \beta $, если $ \xi_{t} < 0 $.
				\end{enumerate} %\vspace{1mm}

                Таким образом в первом случае случайное состояние системы переходит в окрестность элемента детерминированного цикла $ x_{1} = -1 $, а во втором --- остается в окрестности элемента $ x_{k} = \alpha + \beta $.


				\begin{figure}[p] % можно p
				\centering
                	\begin{subfigure}[l]{0.75\textwidth}
                    \centering
                    	\includegraphics[scale=0.7]{fig4_1_stoch_new2.png}
                        \caption{}
                	\end{subfigure}

                    \begin{subfigure}[l]{0.75\textwidth}
                    \centering
                    	\includegraphics[scale=0.7]{fig4_2_stoch_new2.png}
                        \caption{}
                    \end{subfigure}

                    \begin{subfigure}[l]{0.75\textwidth}
                    \centering
                    	\includegraphics[scale=0.7]{fig4_3_stoch_new2.png}%либо scale=0.5 stoch_2
                        \caption{}
                    \end{subfigure}
                    \caption{Аппроксимация функции плотности распределения для \\$ \alpha = 3 $,~$ \varepsilon = 0.01 $: а) цикл периода 3 ($ \beta = -1.25 $); б) цикл периода 4 ($ \beta =~-1.75 $); в) цикл периода 5 ($ \beta = -2.1 $)}
                    \label{fig4_stoch}
				\end{figure}
                %\newpage

			    \begin{figure}[h!]
					\begin{center}
						\includegraphics[width=0.68\textwidth]{fig4_4_2new_stoch.png}
					\end{center}
					\caption{Временной ряд для цикла периода 5 для $ \alpha = 3 $,~$ \varepsilon = 0.01 $ при $ \beta = -2.1 $}
					\label{fig4_4_stoch}		
				\end{figure}

				На рисунке \ref{fig4_4_stoch} такое поведение представлено временным рядом для цикла периода 5 ($ \beta = -2.1 $). Элементы цикла распределены неравномерно, в частности возникают ситуации, когда элемент стохастической траектории на несколько итераций задерживается в окрестности $ \bar{x}_{5} = 0.9 $.


		\newpage

		\subsection{Стохастические феномены}
		\label{stoch_phen}
			В данном разделе, опираясь на описанную ранее технику ФСЧ, изучаются следующие стохастические феномены, наблюдаемые в системе (\ref{stoch_model:eq1}):
			\begin{enumerate}
				\item переходы внутри аттрактора при $ \alpha = 3 $;
				\item переходы между аттракторами при $ \alpha = 8 $:
				\begin{enumerate}
					\item с равновесия на цикл;
					\item с цикла на равновесие;
				\end{enumerate}
				\item генерация большеамплитудных колебаний при $ \alpha = 3 $.
			\end{enumerate}
			\subsubsection{Переходы внутри аттрактора}
			\label{stoch_one_attractor}
            	Как было показано в разделе \ref{approx_disp_density}, под действием шума случайные траектории "задерживаются"\ в окрестности элемента цикла $ x_{k} = \alpha + \beta $.

				На рисунке \ref{fig8_1_stoch} в зависимости от значения интенсивности шума представлена частота попадания стохастических состояний в окрестность одного из элементов цикла для $ \alpha =~3 $,~$ \beta = -1.25 $:
                \begin{itemize}
                	\item синий: $ \bar{x}_{1} = -1 $ --- $ n_{1} $;
                    \item красный: $ \bar{x}_{2} = \frac{\alpha}{2} + \beta = 0.25 $ --- $ n_{2} $;
                	\item зелёный: $ \bar{x}_{3} = \alpha + \beta = 1.75 $ --- $ n_{3} $.
                \end{itemize}

                Видно, что если $ \varepsilon = 0 $,~то $ n_{1} = n_{2} = n_{3} $,~т.е. элементы цикла "посещаются"\ одинаковое количество раз. При увеличении интенсивности шума $ \varepsilon $ происходит скачкообразное изменение значения частот $ n_{1},~n_{2},~n_{3} $,~природа которых описывалась ранее. Здесь важно отметить, что при $ \varepsilon > 0.2 $ значения частот $ n_{1} $ и $ n_{2} $ начинают различаться, т.е. окрестности элементов $ \bar{x}_{1} $ и $ \bar{x}_{2} $ "посещаются"\ не одинаковое количество раз. Такое поведение объясняется слиянием элементов стохастического цикла, и уже невозможно определить, принадлежит состояние окрестности элемента $ \bar{x}_{1} $ или $ \bar{x}_{2} $.

				\begin{figure}[h!]
					\begin{center}
						\includegraphics[width=0.68\textwidth]{fig8_1_stoch_new.png}
					\end{center}
					\caption{Частота попадания точек стохастического цикла для $ \alpha = 3 $ и $ \beta = -1.25 $}
					\label{fig8_1_stoch}		
				\end{figure} %\vspace{1mm}

				На рисунке \ref{fig8_3_stoch} такое поведение хорошо видно на стохастической диаграмме, построенной в зависимости от $ \varepsilon $. Если $ \varepsilon < 0.2 $,~то разделение между элементами еще отчетливо видно, в то время как при $ \varepsilon > 0.2 $ --- уже нет.
				\begin{figure}[h!]
					\begin{center}
						\includegraphics[width=0.68\textwidth]{fig8_3_stoch_new.png}
					\end{center}
					\caption{Стохастическая диаграмма для $ \alpha = 3 $ и $ \beta = -1.25 $}
					\label{fig8_3_stoch}		
				\end{figure} %\vspace{1mm}

				Опираясь на технику ФСЧ (см. раздел \ref{approx_disp_density}), на рисунке \ref{fig8_2_stoch} построена теоретическая плотность распределения для $ \alpha = 3 $,~$ \beta = -1.25 $ при $ \varepsilon = 0.1 $~(синий) и $ \varepsilon = 0.3 $~(красный). Легко заметить, как меняется форма графика плотности распределения. Для $ \varepsilon = 0.1 $ график имеет три отчетливых пика и области между ними, где плотность равна нулю. Для $ \varepsilon = 0.3 $ эти области уже отсутствуют, пики существенно положе и меньше.
				\begin{figure}[h!]
					\begin{center}
						\includegraphics[width=0.68\textwidth]{fig8_2_stoch_new2.png}
					\end{center}
					\caption{Теоретическая плотность распределения для $ \alpha = 3 $ и \\$ \beta = -1.25 $ при $ \varepsilon = 0.1 $~(синий),~$ \varepsilon = 0.3 $~(красный)}
					\label{fig8_2_stoch}		
				\end{figure} %\vspace{1mm}

                Таким образом, в системе наблюдаются индуцированные шумом переходы между элементами цикла.

			%\newpage
			\subsubsection{Переходы между аттракторами}
			\label{stoch_noise_jumps}
            	Далее рассматривается зона параметров $ \alpha $ и $ \beta $,~где детерминированная система имеет два сосуществующих аттрактора.
            	
            	Опираясь на метод доверительных полос, можно предсказывать, при каких сочетаниях параметров системы и интенсивности случайного воздействия будут наблюдаться переходы между аттракторами.
				
                На рисунке \ref{fig5_1_stoch} при $ \alpha = 8 $ для двух значений интенсивности шума $ \varepsilon = 0.1 $ (см. рисунок \ref{fig5_1_stoch}а) и $ \varepsilon = 0.3 $ (см. рисунок \ref{fig5_1_stoch}б) представлены:
                \begin{itemize}
                	\item синий --- устойчивое равновесие и циклы детерминированной модели;
                    \item красный --- неустойчивое равновесие;
                    \item серый --- случайные состояния;
                    \item зелёный --- полоса рассеивания вокруг равновесия.
                \end{itemize}

                Видно, что когда полоса пересекает неустойчивое равновесие, в системе гарантированно наблюдаются переходы с равновесия на цикл.
				\begin{figure}[h!]
					\begin{center}
						\includegraphics[width=0.75\textwidth]{fig5_1_stoch_new.png}
					\end{center}
					\caption{Переходы с равновесия на цикл при $ \alpha = 8 $: а) $ \varepsilon = 0.1 $; \\ б) $ \varepsilon = 0.3 $}
					\label{fig5_1_stoch}		
				\end{figure} %\vspace{1mm}

				\begin{figure}[h!]
					\begin{center}
						\includegraphics[width=0.75\textwidth]{fig5_2_stoch_new.png}
					\end{center}
					\caption{Переходы с цикла на равновесие при $ \alpha = 8 $: а) $ \varepsilon = 0.01 $; \\ б) $ \varepsilon = 0.05 $}
					\label{fig5_2_stoch}		
				\end{figure} %\vspace{1mm}
                Аналогично, используя метод доверительных полос для цикла, можно предсказывать переходы с цикла на равновесие. На рисунке \ref{fig5_2_stoch} при $ \alpha = 8 $ для $ \varepsilon = 0.01 $ (см. рисунок \ref{fig5_2_stoch}а) и $ \varepsilon = 0.05 $ (см. рисунок \ref{fig5_2_stoch}б) показана полоса рассеивания для элемента цикла с наименьшей координатой. Однако даже полосы вокруг одного элемента цикла достаточно, чтобы предсказать переходы с цикла на равновесие.
			
			%\newpage			

			\subsubsection{Генерация большеамплитудных колебаний}
			\label{stoch_fluc}
				В случае, когда в системе (\ref{model:eq1}) существует только одно устойчивое равновесие, посредством ФСЧ можно показать феномен генерации большеамплитудных колебаний. На рисунке \ref{fig6_stoch}а для $ \alpha = 3 $ и $ \varepsilon = 0.2 $ данный феномен показан на диаграмме: как только доверительная полоса (зелёный цвет) пересекает неустойчивое равновесие (красный цвет), стохастические состояния перепрыгивают неустойчивое равновесие и создают большеамплитудные колебания. На рисунке \ref{fig6_stoch}б этот феномен продемонстрирован временными рядами для двух значений интенсивности: $ \varepsilon = 0.2 $ (синий),~$ \varepsilon = 0.05 $ (красный). Как можно заметить, при большой интенсивности возникают спайки.
				\begin{figure}[h!]
					\begin{center}
						\includegraphics[width=0.75\textwidth]{fig6_stoch_new.png}
					\end{center}
					\caption{Генерация большеампитудных колебаний при $ \alpha = 3 $: \\ а) диаграмма для $ \varepsilon = 0.2 $; б) временной ряд при $ \beta = -2.5 $ для $ \varepsilon = 0.2 $ (синий),~$ \varepsilon = 0.05 $ (красный)}
					\label{fig6_stoch}		
				\end{figure}
			
			\newpage

			Критическая интенсивность шума, больше которой для заданных значений параметров $ \alpha $ и $ \beta $ в системе будут наблюдаться переходы между аттракторами, обозначена как $ \varepsilon^{*} $. В данном случае одномерной системы значение $ \varepsilon^{*} $ может быть найдено аналитически:
			\begin{itemize}
				\item равновесие - цикл: $ \varepsilon^{*} = \frac{\sqrt{\gamma (1 - \beta + \sqrt{\gamma})^{4} - 16 \alpha^{2} \gamma}}{3 (1 - \beta + \sqrt{\gamma})^{2}} $;
				\item цикл - равновесие: $ \varepsilon^{*} =\frac{1}{6} \abs{3 + \beta + \gamma} $;
			\end{itemize}
			где $ \gamma = (\beta - 1)^{2} - 4 \alpha $.

			На рисунке \ref{fig7_stoch} представлены графики критической интенсивности для двух значений параметра $ \alpha $. Для $ \alpha = 8 $ на рисунке \ref{fig7_stoch}а синяя линия отвечает за критическую интенсивность для перехода с равновесия на цикл, а красная --- с цикла на равновесие. Для $ \alpha = 3 $ на рисунке \ref{fig7_stoch}б синим цветом представлена критическая интенсивность для возникновения большеамплитудных колебаний. Представленные результаты согласуются с результатами численных экспериментов, изображенных на рисунках \ref{fig5_1_stoch}, \ref{fig5_2_stoch} и \ref{fig6_stoch}. %\vspace{1mm}
			\begin{figure}[h!]
				\begin{center}
					\includegraphics[width=0.75\textwidth]{fig7_stoch_new2.png}
				\end{center}
				\caption{Критическая интенсивность шума для: а) $ \alpha = 8 $; б) $ \alpha = 3 $}
				\label{fig7_stoch}		
			\end{figure} \vspace{1mm}

			\newpage
			\subsubsection{Межспайковые интервалы}
			\label{stoch_spikes}
            	Широко используемой временной характеристикой в нейронной динамике является межспайковый интервал (ISI) $ \tau $. Как правило, рассматривают среднее значение межспайковых интервалов $ m = \langle\tau\rangle $, дисперсию $ D = \langle(\tau - m)^{2}\rangle $ и коэффициент вариации $ C_{\nu} $, который определяется следующим образом: $ C_{\nu} = \frac{\sqrt{D}}{m} $.

				На рисунке \ref{fig9_1_stoch} показаны графики $ m(\varepsilon) $,~$ D(\varepsilon) $,~$ C_{\nu}(\varepsilon) $ для следующих значений параметров: $ \alpha = 3 $ и $ \beta = -2.5 $. В этом случае, как говорилось раннее (см. раздел \ref{stoch_fluc}), возникновение спайков обусловлено феноменом генерации большеамплитудных колебаний. Для стохастически возмущенного равновесия при малых интенсивностях шума, когда спайки не генерируются, средний межспайковый интервал определяется бесконечностью, дисперсия и коэффициент вариации --- нулём. При увеличении интенсивности среднее значение ISI резко уменьшается, что говорит о начале стохастической генерации спайков.
				\begin{figure}[h!]
					\begin{center}
						\includegraphics[width=0.75\textwidth]{fig9_1_stoch_new.png}
					\end{center}
					\caption{Межспайковые интервалы для $ \alpha = 3$ и $ \beta = -2.5$: а) среднее; б) дисперсия; в) коэффициент вариации}
					\label{fig9_1_stoch}		
				\end{figure} %\vspace{1mm}

				На рисунках \ref{fig9_2_stoch} и \ref{fig9_2_1_stoch} показаны графики $ m(\varepsilon) $,~$ D(\varepsilon) $,~$ C_{\nu}(\varepsilon) $ для следующих значений параметров: $ \alpha = 8 $ и $ \beta = -4.7 $. В этом случае (см. раздел \ref{stoch_noise_jumps}) возникновение спайков обусловлено двумя механизмами:
                \begin{itemize}
                	\item переход между элементами цикла;
                    \item переход от равновесия к циклу.
                \end{itemize}

                На рисунке \ref{fig9_2_stoch} начальная точка всегда выбирается на цикле, в отличие от рисунка \ref{fig9_2_1_stoch}, где выбор приходится на равновесие. В силу этого также наблюдаются различия в поведении статистических характеристик. В первом случае спайки наблюдаются даже в отсутствии шума, в то время как во втором случае --- только с некоторого значения интенсивности. Нахождение критических значений интенсивности шума было подробно изложено в разделе \ref{stoch_fluc}.
				\begin{figure}[h!]
					\begin{center}
						\includegraphics[width=0.75\textwidth]{fig9_2_stoch_new.png}
					\end{center}
					\caption{Межспайковые интервалы для $ \alpha = 8$ и $ \beta = -4.7$,~$ x_{0} = -1$: а) среднее; б) дисперсия; в) коэффициент вариации}
					\label{fig9_2_stoch}		
				\end{figure} %\vspace{1mm}

                \begin{figure}[h!]
					\begin{center}
						\includegraphics[width=0.75\textwidth]{fig9_2_1_2stoch.png}
					\end{center}
					\caption{Межспайковые интервалы для $ \alpha = 8$ и $ \beta = -4.7$,~$ x_{0} = -2.2 $: а) среднее; б) дисперсия; в) коэффициент вариации}
					\label{fig9_2_1_stoch}		
				\end{figure}

				\newpage
				На рисунке \ref{fig9_3_stoch} для значений параметров $ \alpha = 8 $ и $ \beta = -4.7 $ изображены временные ряды, показывающие природу образования спайков при двух различных значениях интенсивности шума, когда начальная точка выбирается на равновесии. На рисунке \ref{fig9_3_stoch}а при $ \varepsilon = 0.1 $ состояния довольно продолжительное время локализуются вокруг равновесия, но если случился переход на цикл, обратного перехода не наблюдается. На рисунке \ref{fig9_3_stoch}б при $ \varepsilon = 0.3 $ возникают обратные переходы на равновесие, где состояния задерживаются на некоторое время. Такое поведение объясняет рост среднего значения межспайковых интервалов (см. рисунок \ref{fig9_2_1_stoch}). Если $ \varepsilon > 0.3 $,~то поведение статистических характеристик практически одинаковое, что объяснимо реализацией переходов как с цикла на равновесие, так и обратно. В этом случае начальная точка уже не играет такой большой роли. Стоит отметить, что значения критических интенсивностей находятся в хорошем согласии с найденными теоретически (см. рисунок \ref{fig7_stoch}).
				
                \begin{figure}[H]
					\begin{center}
						\includegraphics[width=0.75\textwidth]{fig9_3_stoch_new.png}
					\end{center}
					\caption{Временные ряды для $ \alpha = 8 $ и $ \beta = -4.7$ при: а)~$ \varepsilon = 0.1 $; \\ б) $ \varepsilon = 0.3 $}
					\label{fig9_3_stoch}		
				\end{figure} %\vspace{1mm}

                Таким образом, во втором разделе работы проведен подробный анализ чувствительности аттракторов. Основываясь на технике ФСЧ, описаны индуцированные шумом феномены.

\newpage
\likechapter{Заключение}

В работе представлены результаты исследования одномерной модели нейронной активности Рулькова, заданной кусочно-гладким отображением, в детерминированном и стохастическом случаях.

В первом разделе для детерминированной модели были найдены параметрические зоны существования равновесий системы. Также была дана классификация возможных динамических режимов, возникающих при различных значениях параметров $ \alpha $ и $ \beta $. С помощью теории критических точек и абсорбирующих интервалов были найдены бассейны притяжения для сосуществующих аттракторов, определены области начальных точек, имеющие различный переходный процесс. Найдены значения управляющих параметров, при которых в системе происходят бифуркации: гомоклиническая бифуркация и бифуркация столкновения с границей. Дано подробное описание этих бифуркаций, особое внимание уделено бифуркации столкновения с границей, как специальному типу бифуркаций для кусочно-гладких отображений. Определены значения параметра $ \alpha $,~при которых система либо имеет единственный устойчивый аттрактор --- равновесие или цикл, либо наблюдается сосуществование двух устойчивых аттракторов.

Во втором разделе проведен подробный анализ стохастической модели. Используя метод функций стохастической чувствительности, изучена чувствительность аттракторов к внешнему возмущению. Найдены зависимости функций стохастической чувствительности от параметров модели. На их основе построены доверительные полосы, описывающие разброс случайных состояний вокруг детерминированных аттракторов. Успешно продемонстрировано согласование эмпирических характеристик распределения случайных состояний системы с теоретическими (плотность и дисперсия). Опираясь на метод доверительных полос, подробно были изучены стохастические феномены, такие как переходы внутри аттрактора, переходы между аттракторами, а также генерация большеамплитудных колебаний. Были найдены значения параметров системы, при которых реализуются соответствующие индуцированные шумом феномены. Также была представлена зависимость критической интенсивности шума от параметров системы. Исследована природа возникновения спайков в системе, а также статистические характеристики межспайковых интервалов: среднее значение, дисперсия, коэффициент вариации.

Работа была представлена на Международной (49-ой Всероссийской) молодежной школе-конференции <<Современные проблемы математики и ее приложений>>. По итогам участия в конференции была отправлена статья в сборник CEUR Workshop Proceedings.
	\newpage
    %\addcontentsline{toc}{likechapter}{СПИСОК ИСПОЛЬЗОВАННЫХ ИСТОЧНИКОВ}
    \bibliographystyle{ugost2008}
	\bibliography{sample}
\end{document} 